\chapter{Summary and conclusions}
\label{ch:conclusion}

\begin{quotation}
  \emph{Mathematics is biology's next microscope, only better; biology
    is mathematics' next physics, only better.}
\begin{flushright}
J.E. Cohen (2004)
%\citep{Cohen04}
\end{flushright}
\end{quotation}

Following the initial sequencing of the human genome \citep{Lander01,
  Venter01}, the understanding of structural and functional
organization of genetic information has extended rapidly with the
accumulation of research data. This has opened up new challenges and
opportunities for making fundamental discoveries about living
organisms and creating a holistic picture about genome
organization. The increasing need to organize the large volumes of
genomic data with minimal human intervention has made computation an
increasingly central element in modern scientific inquiry. It is a
paradox of our time that the historical scale of data in public and
proprietary repositories is only revealing how incomplete our
knowledge of the enormous complexity of living systems is. The
particular challenges in data-intensive genomics are associated with
the complex and poorly characterized nature of living systems, as well
as with limited availability of observations.  It is possible to solve
some of these challenges by combining statistical power across
multiple experiments, and utilizing the wealth of background
information in public repositories. Exploratory data analysis can help
to provide research hypotheses and material for more detailed
investigations based on large-scale genomic observations when little
prior knowledge is available concerning the underlying phenomena;
models that are robust to uncertainty and able to automatically adapt
to the data, can facilitate the discovery of novel biological
hypotheses. Statistical learning and probabilistic models provide a
natural theoretical framework for such analysis.

In this thesis, general-purpose exploratory data analysis methods have
been developed for organism-wide analysis of the human transcriptome,
a central functional layer of the genome. Integrating evidence across
multiple sources of genomic information can help to reveal mechanisms
that could not be investigated based on smaller and more targeted
experiments; this is a central aspect in all contributions.  In
particular, methods have been developed (i) in order to improve
measurement accuracy of high-throughput observations, (ii) in order to
model transcriptional activation patterns and tissue relatedness in
genome-wide interaction networks at an organism-wide scale, and (iii)
in order to integrate measurements of the human transcriptome with
other layers of genomic information. These results contribute to some
of the 'grand challenges' in the genomic era by developing strategies
to understand cell-biological systems, genetic contributions to human
health and evolutionary variation \citep{Collins03}. The computational
experiments in this thesis have been carried out based on publicly
available, anonymized data sets that follow commonly accepted ethical
standards in biomedical research. Open access implementations of the
key algorithms have been provided to guarantee wide access to these
tools and to spark new research beyond the original applications
presented in this thesis.

Methodological extensions and application of the developed algorithms
to new data integration tasks in functional genomics and in other
fields provide a promising line for future studies. The methods
developed in this thesis are readily applicable in genome-wide
screening studies in cancer and potentially other diseases. Increasing
amounts of co-occurring data concerning various aspects of the genome
have become available, including gene- and micro-RNA expression,
structural variation in the DNA, epigenetic modifications and gene
regulatory networks. It is expected that with small modifications the
introduced methodology can be applied to study further associations
between these and other layers of genome organization, as well as
their contributions to human health. The fundamental research
challenges in contemporary genome biology provide a wide array of
applications for statistical learning and exploratory analysis, and a
rich source of ideas for methodological research.

