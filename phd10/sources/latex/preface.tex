\chapter*{Preface}
\addcontentsline{toc}{section}{Preface}
\markboth{PREFACE}{PREFACE}

This work has been carried out at the Neural Networks Research Centre
and Adaptive Informatics Research Centre of the Laboratory of Computer
and Information Science (Department of Information and Computer
Science since 2008), Helsinki University of Technology, i.e., as of
2010 the Aalto University School of Science and Technology. Part of
the work was done at the Department of Computer Science, University of
Helsinki, when I was visiting there for a year in 2005. I am also
pleased to having had the opportunity to be a part of the Helsinki
Institute for Information Technology HIIT. The work has been supported
by the Graduate School of Computer Science and Engineering, as well as
by project funding from the Academy of Finland through the SYSBIO
program and from TEKES through the MultiBio research consortium. The
Graduate School in Computational Biology, Bioinformatics, and Biometry
(ComBi) has supported my participation to scientific conferences and
workshops abroad during the thesis work.

I wish to thank my supervisor, professor Samuel Kaski for giving me
the opportunity to work in a truly interdisciplinary research field
with the freedom and responsibilities of scientific work, and with the
necessary amount of guidance. These have been essential parts of the
learning process.

I would also like to express my gratitude to the reviewers of this
thesis, Professor Juho Rousu and Doctor Simon Rogers for their expert
feedback.

Research on computational biology has given me the excellent
opportunity to work with and learn from experts in two traditionally
distinct disciplines, computational science and genome biology. I am
particularly grateful to professor Sakari Knuutila for his enthusiasm,
curiosity, and personal example in collaboration and daily research
work. Researchers in the Laboratory of Cytomolecular Genetics at the
Haartman Institute have provided a friendly and inspiring environment
for active collaboration during the last years.

My sincere compliments belong to all of my other co-authors, in
particular to Tero Aittokallio, Laura Elo-Uhlgren, Jaakko Hollm�n,
Juha Knuuttila, Samuel Myllykangas and Janne Nikkil�. It has been a
pleasure to work with you, and your contributions extend beyond what
we wrote together. I would also like to thank the former and present
members of the MI research group for working beside me through these
years, as well as for intriguing discussions about science and life in
general. I would also like to thank the personnel of the ICS
department, in particular professors Erkki Oja and Olli Simula, who
have helped to provide an excellent academic research environment, as well as
our secretaries Tarja Pihamaa and Leila Koivisto, and Markku Ranta and
Miki Sirola, who have given valuable help in so many practical matters
during the years.

Science is a community effort. Open sharing of ideas, knowledge,
publication material, data, software, code, experiences and emotions
has had a tremendous impact to this thesis. I will express my sincere
gratitude to the community by continued participation and
contributions. 

I would also like to thank my earliest scientific advisors; Reijo, who
brought me writings about the chemistry of life and helped me to grow
bacteria and prepare space dust in the 1980's, Pekka, who has
demonstrated the power of criticism and emphasized that natural
science has to be exact, Tapio, for the attitude that maths can be
just fun, and Risto, for showing how rational thinking can be applied
also in real life. Thanks also go to my science friends, Manu and
Ville; we have shared the passion for natural science, and I want to
thank you for our continuous and inspiring discussions along the way.
I am grateful to my grandfather Osmo, who shared with me the wonder
towards life, science, and humanities, and was willing to discuss it
all through days and nights when I was a child, questioning himself
the self-evident truths again and again, remaining as puzzled as I
was. And for Alli and Arja, my grandmothers, for their understanding,
and all support and love.

My Friends. With you I have explored other facets of nature, science,
and life... Thank you for staying with me through all these years and
sharing so many aspects of curiosity, exploration and mutual
understanding. 

Finally, I am grateful to my parents and sister, Pipsa, Kari, and
Tuuli. You have accepted me and loved me, supported me on the paths
that I have chosen to follow, and understood that freedom can create
the strongest ties.\\[2mm]

\noindent Cambridge, November 23, 2010\\
\noindent Leo Lahti
